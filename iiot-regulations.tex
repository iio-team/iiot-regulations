\documentclass{article}
\usepackage[utf8]{inputenc}
\usepackage[english]{babel}
\usepackage{hyperref}
\usepackage{graphicx}

\setlength{\parindent}{4em} 


\setlength{\parskip}{1em}


\renewcommand{\baselinestretch}{1.5}

\usepackage{blindtext}

\title{International Informatics Olympiad in Teams \\ Regulations [DRAFT]}

\begin{document}

\maketitle

%-------------------------------------------------------
\section{INTRODUCTION}

The International Olympiad is organized by an International Committee (IC), composed of representatives from countries that are regular members of the competition. Regular members are countries that have been approved by the IC and have hosted the International Olympiad or have committed to hosting it in the future.

The Olympiad was born as a competition between schools and, as such, maintains its core characteristics. In each country, one school is responsible for organizing the national championship by collecting registrations and hosting the in-person national final. This school is called the "leader school." In relation to the IIOT, it has its own delegation, composed of the headmaster and a teacher, who participate in official meetings alongside the president of the national committee, the scientific coordinator, and the representative of the Ministry of Education. A team and its accompanying teacher from the school are automatically admitted to the in-person international final. Each national delegation (headmaster and a teacher from the leader school, national committee president, scientific coordinator, and Ministry representative) is usually represented by a coordinator to facilitate communications and decision-making within the IIOT, especially during contacts throughout the year.

To meet the needs of countries where programming is not a core curriculum subject, or where an organization is appropriate to run the competition instead of a leader school, national organizations that maintain the same characteristics of leader school representation are welcomed, with different representatives as outlined in section five.

Currently, the regular members include Italy, Romania, 
Egypt, Syria, 
Hungary, the United States of America, and Bulgaria. 
Italy and Romania are recognized as the founding 
members of the competition.

Other countries may be invited to become \textbf{regular} members, thereby gaining full membership in the IC, provided they commit to organizing a future edition of the Olympiad. Additionally, the host country of any given edition may invite \textbf{guest} teams from other countries. These guest teams compete unofficially and may be required to pay a participation fee, depending on the circumstances.

English is the official language for all communication, including documents, problem statements, websites, emails, and appeals. The IIOT logo, name, and trophy are the property of the International Committee and may only be used with its formal written permission. They may not be used for commercial purposes under any circumstances.

\section{GOALS}

The International Informatics Olympiad in Teams (IIOT) sets its primary goal as motivating secondary school students to pursue informatics and problem-solving in general. It provides an environment where students can test and demonstrate their abilities in solving computational problems, exchange knowledge and experiences with peers with similar interests, and establish relationships with talented students from other countries. The primary objective of the IIOT is to stimulate the interest of young people in computer science and informatics alongside the Individual Computer Olympiad.

In today's world, teamwork is essential for ensuring the success of any organization, and it is critical for every member of an organization to have the ability to work effectively and collaboratively in teams. In collaborative work, team members can rely on one another's strengths to achieve results greater than the sum of their parts. By working together, they can overcome challenges more effectively while benefiting from diverse ideas and backgrounds that drive the organization forward. In addition, individuals who are more capable of working together with their colleagues benefit from having a better life-work balance and are more productive in their daily tasks.

\section{GENERAL REGULATIONS}

Every team is coordinated by one team leader, who will also register the team in the competition. The only expenditure that each country's delegation must support is the cost of travel to where the competition is organized. Each country’s delegation includes the national selection winners—four students and one teacher—as well as the team from the leader school or second winning team, if applicable—also four students and one teacher. Additionally, the national committee may send up to four members. All other local expenses are the host country's responsibility, which is required to cover them.

Additional persons and observers are welcome, provided they cover the full expenses for their stay and contact the local organizers in advance. The organizers must provide the hardware and software infrastructure required for the competition. Notably, the computer programs must be in the English language. Each team may bring peripheral devices (mouse, keyboard) and use them if they are approved by the scientific committee.

\section{TEAMS}

An eligible contestant is a student enrolled in an institution providing secondary education in the country they are representing. The student must be enrolled at least during the period between September 1st and December 31st of the year before the IIOT final. The student must also not be older than 20 on July 1st of the year of the IIOT final. Exceptionally, students who study outside of their country of nationality may represent the country of their nationality. 

In the case that all team members have a nationality other than their school's country, or the team represents a linguistic minority school, then the team may represent the foreign country if that country allows it. In such cases, the team can only participate in one country's qualifying contest, and they must choose before the start of the preliminary online competitions. 

Each team consists of four students and two reserves if needed. A team can include no more than one contestant who was awarded at the National Individual Olympiad in Programming in the previous year (for practical purposes, the teams’ composition refers to the last available results of the National Individual Olympiad). Students cannot change their teams during the season.  

A team consists of students from the same educational institution. If and only if a national organization has in charge all the management (see the next point in the Regulations: “The International Committee”), students must come from the same institution or region in the mandatory spirit of preserving the format of the teams that compete in the national context. 

Teams attending the International Final are directly selected by the country they represent according to their own rules.  Regular countries must have some form of a national qualifying contest in person.

\section{THE INTERNATIONAL COMMITTEE} 

The International Committee (IC) is the leading and regulatory board of IIOT; it consists of the countries' representatives who are regular members. Each country is represented by a national committee that consists of four people: the representative of the Ministry of Education or another appropriate institution, the scientific coordinator, the headmaster, and a teacher of the leader school. 

The competition was born and must remain a competition between schools, so we consider mainly schools as representatives and managers of each national championship. If and only if programming is not part of the national educational curriculum, a national organization may be in charge of managing the national championship and has the same rights as any other country's leading school, except for having an additional team at the International final, as there is no leader school to coordinate that said team. In this case, the national committee will consist of the president and a qualified member of the national organization, instead of the headmaster and the teacher of the leader school.

Every year during the IIOT competition days, IC members have a round-table meeting to evalute the current edition, planning the next year’s edition, making changes in the Regulations if needed, and addressing the requests of nations wishing to become part of the project. During these meetings, IC members elect the following officials:

\begin{itemize}
    \item The International Secretary, elected by an IC majority vote, on a three-year term and can be removed by a two-thirds IC majority vote during an IC meeting. The International Secretary is responsible for updating the IIOT website, carrying out the countries admission formalities, writing the minutes at the end of the IC meetings, and ensuring the enforcement and integrity of the IIOT regulations.
    \item The International Coordinator is elected by an IC majority vote on a three-year term and can be removed by a two-thirds IC majority vote during an IC meeting. The International Coordinator is responsible for maintaining the IIOT institutional email, ensuring communication and cooperation among the IC members, representing the project internationally, promoting the competition, and establishing links with other National and International Informatics Olympiads in synergy with the IC. 
\end{itemize}

The voting procedure is based on the principle “one country, one vote”; Among others, the following votes require the following definition of a majority:

\begin{itemize}
    \item Selecting the next host requires a simple majority vote in the IC
    \item Any revisions to the IIOT Regulations require a two-thirds majority vote in the IC
    \item Enlarging or reducing the set of regular IIOT members requires a two-thirds majority vote in the IC
\end{itemize}

Every other decision not mentioned here requires a simple majority vote.

IC meetings can also be held online and are open to anyone, provided their presence has a specific and well-defined purpose. In that case, the IC makes the invitation, and the invited people have no voting rights.

\section{THE GENERAL ASSEMBLY} 

The General Assembly (GA) is the organization reuniting representatives from all IIOT nations, both regular and guest nations. The members of the General Assembly are as follows:

\begin{itemize}
    \item A president and a group of experts from the host nation
    \item Team leaders of the participating countries
    \item The delegation of each regular member, as mentioned in the International Committee section
\end{itemize}

The GA determines the cutoffs for the gold, silver, and bronze medals. The voting procedure is based on the principle “one country, one vote,” and votes require a simple majority. 

The GA is active only during that specific IIOT edition. 

\section{THE SCIENTIFIC COMMITTEE} 

%% Outdated

The National Scientific Committee (SC) of each regular participant country of the IIOT consists of scientific experts who could be teachers, university students, or professionals, coordinated by a university referent (usually a professor). The National Scientific Committees become active well before the beginning of the Olympiad and are responsible for selecting and preparing problem proposals and testing and evaluating the solutions of the contestants for their national championships. 

The Scientific Committee of the host country should prepare at least one extra problem proposal for the International Final Competition besides the problems (at least seven) given to the contestants. They will be presented to the International Scientific Committee (consisting of the scientific coordinators) the day before the contest. The International SC has the right to deny the proposal of a problem prepared and proposed by the Scientific Committee of the host country in case of a significant issue (ambiguity, correctness, or other serious reason), hence the reason why the SC of the host country should prepare at least one extra proposal. 

The International SC shall not revise the problems' text unless absolutely necessary. In addition, the International SC should set guidelines for the Syllabus and problem development for the next edition of IIOT.

The National SC can receive help from a Technical Committee, which can help the SC with various technical matters, such as ensuring the contest server (CMS), Virtual Machines, hardware, software, and security solutions to ensure the integrity of the competition. 

One month before the IIOT, the host national Scientific Committee can collect the requests from competing countries. 

\section{COMPETITIONS} 

Each regular member can decide on their national selection procedure independently, but the National Final contest must be onsite and held by March 31st of the year of the IIOT. Each nation must have a dedicated website where the following data must be available:

\begin{itemize}
    \item Data regarding each team: team name, members (names, dates of birth, class attended), school (name and city), following the current legal provisions regarding data protection and collection laws
    \item The schedule of the competitions
    \item The standings of the competition, as it currently stands
\end{itemize}

For instance, the standard and recommended format for the National Competition is as follows: 


\begin{itemize}
    \item Four online preliminary competitions, each lasting three hours and solving at least seven problems.
    \item One national final, organized onsite, lasting at least three hours and solving at least seven problems.
    \item For all rounds, a nationally dedicated platform (usually a CMS instance) must be provided to ensure the automated evaluation of problem solutions.
    \item Each team is given a valid username for the duration of the competition and a different password for each round.
    \item After the four preliminary competitions, the teams who qualified for the National Final are decided according to the sum of the scores of the four rounds. The scores of the national round do not include the scores obtained in any preliminary competition.
\end{itemize}

%% Repetition

While teams are generally required to consist of students representing the same educational institution, in nations where programming is not part of the national curriculum, the students may be enrolled in an organization and come from the same region. While the definition of a region differs from country to country, we broadly consider a region as a city and the surrounding area, with the distance between the locations of the team members being small enough to allow the attendance of the team members in the exact location for each of the qualifying rounds and the national final.

Each coordinating teacher assigned to a team will ensure the proper running of the national competitions, checking the following rules are being respected:

\begin{itemize}
\item that students do not use mobile phones, tablets, or any electronic device; 
\item that students do not consult textbooks or translators; 
\item that the internet connection in the contest rooms is disabled, except for the competition Platform; 
\item that USB ports are disabled; 
\item that communication between teams is not possible. 
\end{itemize}

The first winning team (4 students and a teacher) of each National Final Competition will participate in person at the International Competition. In addition, each regular member nation's second-ranked team (4 students and a teacher) may participate in the International Competition in person. In this case, they will be officially ranked as the other regular teams, but they may have to pay a participation fee for their stay, as mentioned in the previous section. The Nation Leader Schools of the regular participants participate for free with one more team (4 students and a teacher) as “a special guest team.” 

The host country can have no more than three new additional teams as guests, no more than five teams in total. In addition, the teams representing the guest nations participate in the International Competition in person. 

The International Competition, in person, lasts four hours and involves solving at least seven problems. Students should enter the room half an hour before the beginning of the contest to check that everything works properly. Nobody should enter the technical room dedicated to the contest except the host country's scientific and technical staff. 

%% Outdated

Contestants may submit written questions to the host country's scientific committee concerning the formulation and interpretation of the problems during the initial period of each competition round. Solving the problems requires no special hardware requirement or software packages (e.g., graphic packages). 

The entire communication between the IIOT committees and contestants is written. In the exceptional case of having online teams as guests in the International Competition, the recording of their computers' activity for online teams must be sent within 3 hours after the end of the contest. The organizing committee reserves its right to check the recordings should any suspicions occur.

\section{PROBLEMS} 

% Py?

All problem statements are given in English. Students must write computer programs that solve the problems (a minimum of 7 problems) assigned to them within the time and memory constraints mentioned in the statement and on the contest platform.

The programming languages that students can use are C, C++, and Pascal. 

The problems given at IIOT include, among other topics, sorting and searching, greedy algorithms, recursion, dynamic programming, graphs, trees, lists, data structures, etc. A complete Syllabus will be available on the competition website. 

\section{EVALUATION} 

% Outdated. Changed by authors to CMS.

The solutions sent by each team are graded on an ongoing basis by the competition server (CMS). A team's score on a given problem is the maximum of the scores obtained on each subtask.

If a team leader does not accept the evaluation results, they may appeal to the International Scientific Committee. 

\section{RESULTS AND PRIZES} 

The General Assembly determines the gold, silver, and bronze medal cutoffs. The proportion of these medals should be approximately 1:2:3. 

At least half of the contestants should receive medals. Each contestant receives a certificate of participation. Each participant team, both regular and guest teams, has the right to be awarded according to its results. The official ranking will present only the results of the regular participants. The results of the guest teams will not be added to the official ranking. 

The GA may disqualify a contestant for unethical behavior during the competition or outside of it, including, but not limited to, behavior that may ruin individuals, countries, or the reputation of the IIOT. 

\section{HOST COUNTRY} 

A host country willing to organize an IIOT edition in a given year in their own country has to announce its intention at least one year in advance (at the latest, during the IIOT competition of the previous year). The IC selects the next host during its annual meeting. The host country proposes at least two dates for the next IIOT edition to the other IC members, which the International Committee will vote for to avoid any overlap with national exams and other competitions; the latest internationals can start is May 31st. 

The host country communicates the official dates to all the participating countries by January of the IIOT edition year with all the needed information: location, visa requirements, and invitations. The host country also sets up a dedicated website by the beginning of January of the current year of IIOT. 

At a minimum, the following information must be available:

\begin{itemize}
    \item the schedule
    \item the list of software programs and the environment the students will have in the competition (operating system, editors, IDEs, compilers, etc.)
    \item information regarding the accommodation (address, details, and pictures of the planned accommodation)
    \item information about visa acquirement
    \item regulations
\end{itemize}

The competition program should include opening and award ceremonies, a practice round and the proper competition, and social and cultural activities. At the end of the event, the host country provides certificates of attendance for all the participants: students, teachers, and delegates.

The award ceremony should be part of the closing ceremony. It also involves passing the official trophy to the next year's hosts. The host country decides the fees that must be paid by the second team of regular members, the guest country teams, and the generic guests (accompanying persons and observers). Fees for the second national team and for guest country teams should be smaller than fees for generic guests. 

% Questions regarding "hotel stars" requirements

The host country will take into account any personal, religious, or cultural dietary requirements as much as possible. There should also be some minimal conditions about the quality of the accommodation: at a minimum, the hosting facility should be rated with three stars or the national equivalent of a three-star hotel out of a five-star system, the rooms should be of proper quality, and adults should have a single room. 

%% Repetition

The host country should organize Special cultural activities for the guests (additional persons and observers) during the practice and the competition. 

The host country must provide an adequate number of PCs with the same software configuration as mentioned on the contest website before the competition. The host country must also include a sufficient number of spare computers in case of any emergency. USB drives and Internet connections must be disabled, with the exception of the contest platform. Computers must be able to read PDF files. 

Documentation of the programming languages allowed should be available during the competition. The PCs used during the practice session should be cleaned of data; teams should not receive the same PC during the contest and the practice. The problem author and one of the collaborators should be reachable during the contest to answer possible questions about the problems the teams may address. 

\section{REGULATIONS REVISION}

% Repetition, guest country suggestion deadline?

Regular members may propose regulation revisions to the International Committee with at least one month before the beginning of the IIOT competition. Guest members may also submit suggestions. IC members decide on any potential revision of the regulations during the annual meeting. Any revision requires a two-thirds majority vote and becomes effective from the following year’s edition of the IIOT. 


\begin{center}
\textbf{Signed by the National Delegations of the IIOT in Order of Enrollment as Regular Participants}
\end{center}

\noindent National Referent Nadia Amaroli of Italy

\href{https://avbo.edu.it/}{Leader School Istituti di Istruzione Superiore Aldini Valeriani}

\noindent National Referent Daniela Neamțu of Romania

\href{http://cni.nt.edu.ro/new/}{Leader School Colegiul Național de Informatică}

\noindent National Referent Eslam Wageed of Egypt

\href{https://aast.edu/en/}{Arab Academy for Science and Technology Regional Informatics Center}

\noindent National Referent Ammar Alnahhas of Syria

\href{https://dca-net.org/}{Syrian Distinction and Creativity Agency}

\noindent National Referent László Nikházy of Hungary

\href{https://fazekas.hu/}{Leader School Budapesti Fazekas Mihály Gyakorló Általános Iskola és Gimnázium}

\noindent National Referent Dylan Karpf of the United States of America

\href{https://www.usiot.net/}{United States of America Informatics Olympiad in Teams}

\noindent National Referent Iliyan Yordanov of Bulgaria

\href{https://infos.infosbg.com/}{Bulgarian National Informatics Committee}




\end{document}
